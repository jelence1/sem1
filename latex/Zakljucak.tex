\chapter{Zaključak}

Provedeno je istraživanje mogućnosti razvojnog sustava ESP32-C3-DevKitM-1 u razvoju Bluetooth aplikacija. Opisana su tehnička svojstva razvojnog sustava te je detaljnije opisan \textit{Bluetooth Low Energy} (BLE) protokol. Navedena su aplikacijska programska sučelja koja sustav podržava te su demonstrirane njihove mogućnosti demo aplikacijama. Mobilnom aplikacijom \textit{nRF Connect} ispitana su svojstva razvojnog sustava kao Bluetooth uređaja. Naposljetku su opisana ograničenja modula pri izradi Bluetooth aplikacija.

Ovaj seminar ponajprije služi kao uvid u mogućnosti razvojnog sustava ESP32-C3-DevKitM-1. Analiza programskih sučelja pokazuje da razvojni sustav nudi velik broj značajki koje se, koristeći analizirana sučelja, mogu jednostavno koristiti i implementirati. Isto tako, analiza jačine emitiranog Bluetooth signala pokazuje da uređaj dobro funkcionira na udaljenostima dovoljnima za pokrivanje kućanstva, no isto tako nudi mogućnost stvaranja mreže za pokrivanje veće površine. Jedino ozbiljnije ograničenje sustava jest nekompatibilnost s uređajima koji rade isključivo na temelju klasičnog Bluetootha. Ipak, provedenim istraživanjem dokazano je da modul ESP32-C3 opravdava svoju nadmoć u IoT primjeni. 

\eject