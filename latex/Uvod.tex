\chapter{Uvod}

Pojam „internet stvari“ neizostavan je u današnjem razvoju bežičnih i pametnih uređaja. IoT (engl. \textit{Internet of things}) krovni je naziv koji obuhvaća milijune uređaja, odnosno „stvari“ povezanih na internet, koji pohranjuju i razmjenjuju podatke s drugim uređajima i sustavima također povezanih na internet. \cite{what_is_iot} 

Za razvoj IoT uređaja potrebni su mikrokontroleri s mogućnošću bežičnog povezivanja. Iako je serija Raspberry Pi \textit{single-board} računala zaklade \textit{Raspberry Pi Foundation} pri vrhu popularnosti u razvoju IoT proizoda, serija ESP32 mikrokontrolera tvrtke \textit{Espressif} pruža ozbiljnu konkurenciju zbog niske potrošnje, visoke otpornosti na temperature, te najvažnije, jednostavnom bežičnom povezivosti. \cite{rasp_pi} \cite{rasp_pi_esp} Jedan takav čip je ESP32-C3, koji pruža Wi-Fi i Bluetooth povezivanje. Čip je integriran u nekoliko različitih modula, koji su pak dio razvojnih sustava koje proizvodi \textit{Espressif}. Za izradu ovog rada odabran je razvojni sustav ESP32-C3-DevKitM-1.

Ovaj seminar analizira mogućnosti koje pruža ESP32-C3-DevKitM-1 u razvoju Bluetooth programskih rješenja. Opisana sui ispitana  programska aplikacijska sučelja (engl. \textit{Application Programming Interface - API}) koje modul podržava. Također, razložena su i ograničenja sustava pri korištenju Bluetooth protokola.

Rad je podijeljen u cjeline kako slijedi. Drugo poglavlje „\textit{Razvojni sustav ESP32-C3-DevKitM-1}“ opisuje osnovne karakteristike korištenog razvojnog sustava kao ciljane hardverske platforme te su opisane najvažnije značajke BLE protokola. U trećem poglavlju „\textit{Aplikacijska programska sučelja}“ opisani su API-ji koji se mogu koristiti uz razvojni sustav. U četvrtom poglavlju „\textit{Usporedba API-ja i ograničenja sustava}“ uspoređena su ranije opisana aplikacijska sučelja te su navedena ograničenja razvojnog sustava u izradi Bluetooth aplikacija.

\eject